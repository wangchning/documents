%========宏包加载
\usepackage[a4paper,margin=2.5cm]{geometry}%设置页面格式,上下左右各留2.5cm
\usepackage[fancyhdr]{ctexcap} %中文
%\hypersetup{colorlinks=false}
\usepackage{amsmath,amssymb,amsthm} %公式
\usepackage{booktabs,longtable} %表
\usepackage{graphicx,float} %图
\usepackage{tikz}%绘图宏包
\usepackage{cite}%参考文献
\usepackage{setspace}%设定行距
\usepackage{lastpage} %记录最后一页的页码
\usepackage{wrapfig} %图文混排
\usepackage{titletoc} %
\usepackage{ccaption} %支持图表中英双标题
\usetikzlibrary{calc,intersections,through,backgrounds}%调用Tikz的libray

%========字体
\setCJKfamilyfont{huawen}{STXihei}
\setCJKfamilyfont{hwzhs}{STZhongsong}
\newcommand\xihei{\CJKfamily{huawen}}%华文细黑
\newcommand\zhongsong{\CJKfamily{hwzhs}}%华文中宋

%========标题格式
\CTEXsetup[format={\zihao{-4}\heiti\bf}]{section}%章节标题居左,四号,黑体
\CTEXsetup[format={\zihao{-4}\heiti\bf}]{subsection}%章节次标题居左,小四号,黑体
\CTEXsetup[format={\zihao{-4}\heiti\bf}]{subsubsection}
\newcommand\msection[1]{\section*{#1}\addcontentsline{toc}{section}{#1}}
%========页面格式
\pagestyle{fancy}
\renewcommand{\headrulewidth}{0pt}
\rhead{}\lhead{}
\fancyfoot[C]{\kaishu\small 第\thepage 页\quad 共~\pageref{LastPage} 页}


\linespread{1.5}%1.5倍行距
\setlength\lineskiplimit{0pt}\setlength\lineskip{1ex}%当间距太小时,跳一个ex

%========目录
\setcounter{tocdepth}{3}%目录深度设为3级
\newcommand\mycontents{\pagestyle{plain}\pagenumbering{Roman}\tableofcontents\newpage}
\renewcommand\contentsname{\centerline{\zihao{3} 目录}}

\makeatletter
\def\fnum@table#1{\tablename\nobreakspace\thetable\hspace{1em}}%去除表标题中的':'
\def\fnum@figure#1{\figurename\nobreakspace\thefigure\hspace{1em}}%去除图标题中的':'
\renewcommand*\l@section{\@dottedtocline{1}{0em}{1.0em}}
\makeatother
%========重定义命令
\def\emph#1{{\heiti #1}}

%========数学公式
\setlength\mathsurround{0.5ex}%文本公式前后留下空白
\makeatletter
\renewcommand\normalsize{\@setfontsize\normalsize\@xpt\@xiipt\abovedisplayskip8\p@\@plus3\p@\@minus5\p@\abovedisplayshortskip\z@\@plus3\p@\belowdisplayshortskip5\p@\@plus3\p@\@minus7\p@\belowdisplayskip\abovedisplayskip\let\@listi\@listI}%行间公式上下
\@addtoreset{equation}{section}
\renewcommand\theequation{\oldstylenums{\thesection}.\oldstylenums{\arabic{equation}}}%公式按章节编号
\makeatother

\allowdisplaybreaks %允许公式跨页

%========定理环境
\newtheoremstyle{mystyle}{5pt}{5pt}{\songti}{2em}{\heiti\bf}{}{1em}{}
%style名称 %环境前空白 %环境后空白 %主字体 %theorem名前的缩进 %theorem字体 % %theorem后缩进
\theoremstyle{mystyle}
\newtheorem{definition}{定义}[section]
\newtheorem{theorem}{定理}[section]
\newtheorem{corollary}{推论}[section]
\newtheorem{property}{问题}[section]
\newtheorem{proposition}{猜测}[section]
\newtheorem{lemma}{引理}[section]
\newtheorem{example}{例}[section]
\newenvironment{key}{\par\indent{\heiti 解~~}}{\hspace*{\fill} $\Box$\par}
\renewenvironment{proof}{\par\indent{\heiti 证明~~}}{\hspace*{\fill}$\Box$\par}

%========常用命令
\newcommand*\circled[1]{\hspace{1pt}\tikz[baseline=(char.base)]\node[very thin,shape=circle,draw,inner sep=.5pt,minimum size=2pt](char){#1};\hspace{1pt}}
\renewcommand\mod[1]{\ (\text{mod}\;#1)}
\newcommand{\upcite}[1]{$\!\!\!^{\mbox{\scriptsize\cite{#1}}}$}

%========论文封面
\def\maketitle{%
\begin{titlepage}\ttfamily
\begin{center}
\includegraphics[width=2in]{swu.jpg}
\par\vspace{3em}
\zihao{1} {\heiti 本科毕业论文(设计)}
\vfil
\zhongsong\zihao{2} 题~~~~目\ \underline{\mbox{\title}}
\vfil
\begin{spacing}{2.0}%二倍行距
\zihao{-2}\ziju{.2}
\par\makebox[1.1in][s]{学院}~~\underline{\makebox[2.4in][c]{\college}}
\par\makebox[1.1in][s]{专业}~~\underline{\makebox[2.4in][c]{\major}}
\par\makebox[1.1in][s]{年级}~~\underline{\makebox[2.4in][c]{\grade}}
\par\makebox[1.1in][s]{学号}~~\underline{\makebox[2.4in][c]{\studentID}}
\par\makebox[1.1in][s]{姓名}~~\underline{\makebox[2.4in][c]{\author}}
\par\makebox[1.1in][s]{指导老师}~~\underline{\makebox[2.4in][c]{\don}}
\par\makebox[1.1in][s]{成绩}~~\underline{\makebox[2.4in][c]{}}
\end{spacing}
\vfil\zihao{3}\date\par\vspace{-4em}
\end{center}\end{titlepage}\newpage}

%========中文标题
\def\makezhtitle{%
\pagestyle{fancy}\setcounter{page}{1}\pagenumbering{arabic}
\begin{center}
    \heiti\zihao{3}\title\ 
    \par\hfill\par
    \zihao{-4}\songti \author\ 
    \par\zihao{5} 西南大学,重庆 400715
\end{center}
\par\hfill\par
}

%========英文标题
\def\makeentitle{%
\par\hfill\par
\begin{center}
    \zihao{4}\textbf{\entitle}
    \par\vspace*{1.7em}
    \zihao{5} \enauthor\ 
    \par\zihao{-5} School of Mathematics and Statistics, Southwest University, Chongqing  400715, China
\end{center}}

%========中文摘要
\newenvironment{abstrace}
{\maketitle\mycontents\makezhtitle\zhongsong\zihao{5} 摘要:\fangsong}
{\addcontentsline{toc}{section}{摘要}}
\def\keywords{\par\zhongsong 关键词:\songti}

%========英文摘要
\newenvironment{enabstrace}
{\addcontentsline{toc}{section}{Abstract}\makeentitle
\def\keywords{\par\noindent\textbf{Key words: }}
\par\vspace*{1.5em}\noindent\zihao{5}\textbf{Abstract:}}
{\par\pagebreak[4]\hspace{1em}}

\newenvironment{mybib}
{\begin{thebibliography}{99}
\setlength\itemsep{-2mm}\zihao{5}
\addcontentsline{toc}{section}{参考文献}}
{\end{thebibliography}}

