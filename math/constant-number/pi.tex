\documentclass{article}
\usepackage{amsmath,amssymb,tikz,mathrsfs}
\usetikzlibrary{calc,intersections,backgrounds}
\usepackage[margin=3.5cm]{geometry}
\usepackage[UTF8,space]{ctex}
\usepackage[thmmarks]{ntheorem}
\newtheorem{Definition}{定义}[section]
\newtheorem{Theorem}[Definition]{定理}
\newtheorem{Lemma}{引理}
\newtheorem{Guess}{猜想}[section]
\newtheorem{Proposition}{命题}[section]
\newtheorem{Deduction}[Proposition]{推论}
\theoremstyle{nonumberplain}\theorembodyfont{\normalfont}\theoremsymbol{$\Box$}
\newtheorem{Proof}{证:}
\begin{document}
$$\frac2\pi=\frac{\sqrt2}2\times\frac{\sqrt{2+\sqrt2}}2\times
\frac{\sqrt{2+\sqrt{2+\sqrt2}}}2\times\cdots
~~\hbox{(韦达1593)}$$
$$\frac\pi2=\frac{2\times2}{1\times3}\times\frac{4\times4}{3\times5}
\times\frac{6\times6}{5\times7}\times\cdots~~\hbox{(沃利斯1650)}$$

$$\frac\pi4=4\arctan\frac15-\arctan\frac1{239}~~\hbox{(Martinl1706)}$$

$$\frac\pi4=1-\frac13+\frac15-\frac17+\frac19-\cdots~~\hbox{(Leibniz)}$$

$$\frac\pi4=\frac12-\frac1{3\times2^3}+\frac1{5\times2^5}-
\frac1{7\times{2^7}}+\cdots~~\hbox{(Euler)}$$

$$\frac4\pi=1+\cfrac1{2+\cfrac9{2+\cfrac{25}{2+\cfrac{49}{2+\cdots}}}}~~\hbox{(BrownCouro)}$$

$$\frac\pi6=\frac12+\frac1{2\times3\times2^3}+
\frac{1\times3}{2\times4\times5\times2^5}+\frac{1\times3\times5}
{2\times4\times6\times7\times2^7}+\cdots~~\hbox{(Newton)}$$

$$\frac\pi4=3\arctan\frac14+\arctan\frac5{99}~~\hbox{(Herdon)}$$

$$\sqrt\pi=\int_{-\infty}^\infty e^{-x^2}dx$$

$$\frac{\pi^2}6=1+\frac1{2^2}+\frac1{3^2}+\frac1{4^2}+\cdots~~\hbox{(Euler)}$$

$$\pi=\frac{426880\sqrt{10005}}{\sum_{n=0}^\infty\displaystyle\frac{(6n)!\,
(545140134n+13591409)}{(n!)^3(3n)!\,(-640320)^{3n}}}~~\hbox{(Chudnovsky)}$$
\end{document}
