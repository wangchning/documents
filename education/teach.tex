\usepackage{amsmath,amssymb,xfrac,txfonts,tikz,tasks}
\usetikzlibrary{calc,intersections,through,backgrounds,angles,shapes}
\usepackage{ctex}
%\usepackage[paperheight=6in,paperwidth=4.5in,margin=1cm]{geometry}%kindel专用
%\usepackage[a4paper,margin=2cm]{geometry}%普通
\usepackage[includemp=true,marginparsep=.5cm,marginparwidth=3cm,left=2.5cm,right=2cm]{geometry}%带有旁注
\setlength\mathsurround{0.5ex}
\linespread{1.75}%调整行距
\newcommand\lines[1][1.2]{\,\underline{\mbox{\hspace{#1cm}}}\,}% 填空题的横线
\newcommand\brackets[1][2]{\nolinebreak\hfill\mbox{~(\hspace{#1em})}\\}% 选择题的括号
\newcommand{\question}[1][]{\par\vspace{1ex}\noindent\refstepcounter{numberi}\textbf{\thenumberi.}\ensuremath{#1}}\newcounter{numberi}[subsection]% 每道小题自动编号
\newcommand\quson{\\ \hspace*{1em}\refstepcounter{numberii}\thenumberii)~}\newcounter{numberii}[numberi]% 每道小题自动编号
\newcommand\choice[5][4]{\vspace*{-1em}\begin{tasks}(#1)\task$#2$\task$#3$\task$#4$\task$#5$\end{tasks}\vspace*{-1.2em}}%选择题排版之数学模式
\newcommand\choicex[5][4]{\vspace*{-1em}\begin{tasks}(#1)\task#2\task#3\task#4\task#5\end{tasks}\vspace*{-1.2em}}%选择题排版之数学模式
\newcommand\set[1]{\lbrace\ensuremath{#1}\rbrace}
\newcommand\setx[1]{\lbrace#1\rbrace}
\newcommand\bfR{{\bm R}}
\newcommand\bfN{{\bm N}}
\newcommand\bfZ{{\bm Z}}
\newcommand\bfQ{{\bm Q}}
\newcommand\bfC{{\bm C}}
\def\bfa{{\bm a}}
\def\bfb{{\bm b}}
\def\bfc{{\bm c}}
\def\bfd{{\bm d}}
\def\bfe{{\bm e}}
\def\bfzero{{\bm 0}}
