\documentclass{article}
\usepackage{amsmath,amssymb,xfrac,txfonts,tikz,tasks}
\usetikzlibrary{calc,intersections,through,backgrounds,angles,shapes}
\usepackage{ctex}
%\usepackage[paperheight=6in,paperwidth=4.5in,margin=1cm]{geometry}%kindel专用
%\usepackage[a4paper,margin=2cm]{geometry}%普通
\usepackage[includemp=true,marginparsep=.5cm,marginparwidth=3cm,left=2.5cm,right=2cm]{geometry}%带有旁注
\setlength\mathsurround{0.5ex}
\linespread{1.75}%调整行距
\newcommand\lines[1][1.2]{\,\underline{\mbox{\hspace{#1cm}}}\,}% 填空题的横线
\newcommand\brackets[1][2]{\nolinebreak\hfill\mbox{~(\hspace{#1em})}\\}% 选择题的括号
\newcommand{\question}[1][]{\par\vspace{1ex}\noindent\refstepcounter{numberi}\textbf{\thenumberi.}\ensuremath{#1}}\newcounter{numberi}[subsection]% 每道小题自动编号
\newcommand\quson{\\ \hspace*{1em}\refstepcounter{numberii}\thenumberii)~}\newcounter{numberii}[numberi]% 每道小题自动编号
\newcommand\choice[5][4]{\vspace*{-1em}\begin{tasks}(#1)\task$#2$\task$#3$\task$#4$\task$#5$\end{tasks}\vspace*{-1.2em}}%选择题排版之数学模式
\newcommand\choicex[5][4]{\vspace*{-1em}\begin{tasks}(#1)\task#2\task#3\task#4\task#5\end{tasks}\vspace*{-1.2em}}%选择题排版之数学模式
\newcommand\set[1]{\lbrace\ensuremath{#1}\rbrace}
\newcommand\setx[1]{\lbrace#1\rbrace}
\newcommand\bfR{{\bm R}}
\newcommand\bfN{{\bm N}}
\newcommand\bfZ{{\bm Z}}
\newcommand\bfQ{{\bm Q}}
\newcommand\bfC{{\bm C}}
\def\bfa{{\bm a}}
\def\bfb{{\bm b}}
\def\bfc{{\bm c}}
\def\bfd{{\bm d}}
\def\bfe{{\bm e}}
\def\bfzero{{\bm 0}}

\begin{document}
学习圆锥曲线的秘诀:\emph{见多识广}。
\section{圆锥曲线的性质}
    \question 椭圆的圆周率$e=c/a$的取值范围是$0\le e<1$, $e=0$时,成为圆;$e$越接近于$1$椭圆越扁;$e=\sqrt{1-\dfrac{b^2}{a^2}}$。

    \question 双曲线的离心率$e=c/a$的取值范围的$1<e$, 当$e$接近 1 时双曲线的开口越小。$e=\sqrt{1+\dfrac{b^2}{a^2}}$。

    \question 准线方程$x=\pm \dfrac{a^2}{c}$(适用于椭圆、双曲线)

    \question 弦长公式 $d=\sqrt{1+k^2}|x_1-x_2|=\sqrt{(1+k^2)[(x_1+x_2)^2-4x_1x_2]}=\sqrt{1+\dfrac{1}{k^2}}|y_1-y_2|$ (适用于所有曲线)

    \question 通径长$2b^2/a$(适用于椭圆、双曲线,抛物线的通径长为$2p$)

    \question 椭圆焦半径公式:$|MF_1|=a+ex_0, |MF_2|=a-ex_0,$ 其中$M(x_0,y_0)$是椭圆上任意一点.

    \question 双曲线$\dfrac{x^2}{a^2}+\dfrac{y^2}{b^2}=1(a>0,b>0)$的焦半径公式:(1)当$M(x_0,y_0)$在右支上时$|MF_1|=ex_0+a, |MF_2|=ex_0-a$;(2)当$M(x_0,y_0)$在左支上时$|MF_1|=-ex_0+a, |MF_2|=-ex_0-a$.

    \question 椭圆焦点角形的面积为$S_{\triangle F_1PF_2}=b^2\tan\frac{\gamma}{2}$其中$\angle F_1PF_2=\gamma$.(改正切为余切即可用于双曲线)

    \question 若$P(x_0,y_0)$在椭圆上,则过$P$的椭圆的切线方程是$\dfrac{x_0x}{a^2}+\dfrac{y_0y}{b^2}=1$.

    \question $AB$是椭圆$\dfrac{x^2}{a^2}+\dfrac{y^2}{b^2}=1$的不平行于对称轴的弦,$M(x_0,y_0) $ 为$AB$ 的中点,则 $k_{OM}\cdot k_{AB}=-b^2/a^2$.

    \question $AB$是双曲线$\dfrac{x^2}{a^2}-\dfrac{y^2}{b^2}=1$的不平行于对称轴的弦,$M(x_0,y_0) $ 为$AB$ 的中点,则 $k_{OM}\cdot k_{AB}=b^2/a^2$.

    \question 椭圆参数方程$$\begin{cases}x=a\cos\theta\\y=b\sin\theta\end{cases}$$

    \question 抛物线焦半径公式$|AF|=x_0+\dfrac{p}{2}$ 过焦点的弦长$|AB|=x_1+x_2+p$

    \question 设过抛物线$y^2=2px(p>0)$的焦点$F$的直线$l$与抛物线交于$A(x_1,y_1),B(x_2,y_2)$,直线$OA$与$OB$ 的斜率分别为$k_1,k_2$,直线$l$的倾斜角为$\alpha$,则有$y_1y_2=-p^2$, $x_1x_2=\dfrac{p^2}{4}$, $k_1k_2=-4$, $|OA|=\dfrac{p}{1-\cos\alpha}$, $|OB|=\dfrac{p}{1+\cos\alpha}$, $|AB|=x_1+x_2+p$, $|AB|=\dfrac{2p}{{\sin}^2\alpha}$.

\section{定义的运用}
    \question $F$是椭圆$\dfrac{x^2}{4}+\dfrac{y^2}{3}=1$的右焦点,$A(1,1)$为椭圆内一定点,$P$为椭圆上一动点。$|PA|+|PF|$ 的最小值为\lines %要用第二定义,可改为$|PA|+2|PF|$

    \question 设$F_1,F_2$分别是椭圆$\dfrac{x^2}{25}+\dfrac{y^2}{16}=1$的左、右焦点,$P$ 为椭圆上一点,$M$ 是$F_1P$ 的中点,$|OM|=3$,则$P$点到椭圆左焦点的距离为\lines.

    \question 椭圆$\dfrac{x^2}{12}+\dfrac{y^2}{3}=1$的焦点为$F_1$和$F_2$,点$P$在椭圆上.如果线段$PF_1$ 的中点在$y$轴上,那么$|PF_1|$ 是$|PF_2|$的\lines 倍

 \question $\triangle ABC$中,$B(-5,0),C(5,0)$,且$\sin C-\sin B= \sin A$,求点$A$的轨迹方程。

    \question 抛物线$C: y^2=4x$上一点$P$到点$A(3,4)$与到准线的距离和最小,则点$P$的坐标为\lines

    \question[2014全国 ]已知抛物线$C: y^2=8x$的焦点为$F$,准线为$l$ ,$P$是$l$ 上一点,$Q$是直线$PF$ 与$C$ 的一个交点,若$\overrightarrow{PF}=4\overrightarrow{FQ}$ ,则$|QF|=$\lines

    \question[*]定长为$3$的线段$AB$的两个端点在$y=x^2$上移动,$AB$中点为$M$,求点$M$ 到$x$ 轴的最短距离。%

\section{离心率}
    %求离心率主要有以下几个方向:
%        \begin{itemize}
%          \item 算出$a,c$
%          \item 求出$a,b,c$之间的任意一个比例
%          \item 找到$a,b,c$间的一个齐次方程,变成离心率$e$的方程
%        \end{itemize}

    \question 若$P$是以$F_1,F_2$为焦点的椭圆$\dfrac{x^2}{a^2}+\dfrac{y^2}{b^2}=1(a>b>0)$ 上的一点,且$\overrightarrow{PF_1}\cdot\overrightarrow{PF_2}=0$, $\tan\angle PF_1F_2=\dfrac12$,则此椭圆的离心率为\lines

    \question 椭圆$\dfrac{x^2}{a^2}+\dfrac{y^2}{b^2}+1(a>b>0)$的两顶点为$A(a,0), B(0,b)$, 且左焦点为$F$, $\triangle FAB$ 是以角$B$为直角的直角三角形,则椭圆的离心率$e$为\lines

    \question 已知$F_1(-c,0),F_2(c,0)$为椭圆$\dfrac{x^2}{a^2}+\dfrac{y^2}{b^2}=1(a>b>0)$ 的两个焦点,$P$ 为椭圆上一点且$\overrightarrow{PF_1}\cdot \overrightarrow{PF_2}=c^2$,则此椭圆 离心率的取值范围是\lines.

    \question 已知$F_1,F_2$是双曲线$\dfrac{x^2}{a^2}-\dfrac{y^2}{b^2}=1(a>0,b>0)$的两焦点,以线段$F_1F_2$为边作正三角形$MF_1F_2$,若边$MF_1$的中点$P$在双曲线上,则双曲线的离心率为\lines

\section{过焦点的弦或直线}
%焦半径
    \question 过抛物线$y^2=4x$的焦点$F$的直线交抛物线于$A,B$ 两点,点$O$ 是原点,若$|AF|=3$ , 则$\triangle AOB$ 的面积为\lines

     \question 设抛物线$y^2=4x(p>0)$的焦点为$F$,经过$F$的直线交抛物线于$A,B$两点,点$C$在抛物线的准线上,且$BC\varparallel x$ 轴,证明直线$AC$经过原点。
\section{中点问题}
    %中点弦的斜率公式
    \question[2014江西 ]过点$M(1,1)$ 作斜率为$-\dfrac12$的直线与椭圆$C: \dfrac{x^2}{a^2}+\dfrac{y^2}{b^2}=1(a>b>0)$相交于$A,B$ ,若 $M$是线段$AB$ 的中点,则椭圆$C$ 的离心率为\lines.

    \question 已知双曲线$E$的中心为原点,$F(3,0)$是$E$的焦点 ,过$F$的直线$l$与$E$相交于$A,B$两点,且$AB$的中点为$N(-12,-15)$,则$E$的方程\lines

    \question 已知直线$l$和双曲线$\dfrac{x^2}{a^2}-\dfrac{y^2}{b^2}=1(a>0,b>0)$及其渐近线的交点从左到右依次为$A,B,C,D$。 求证:$|AB|=|CD|$ 。

    \question[2014辽宁 ]已知椭圆$C: \dfrac{x^2}{9}+\dfrac{y^2}{4}=1$, 点$M$与$C$的焦点不重合,若$M$ 关于$C$ 的焦点的对称点分别为$A,B$,线段$MN $的中点在$C$ 上,则$|AN|+|BN|=$\lines



\section{圆锥曲线与圆}
    \question 设 $F1, F2$ 分别是椭圆 $\dfrac{x^2}{4}+y^2=1$ 的左、右焦点,$P$ 是第一象限内该椭圆上的一点,且 $PF_1\perp PF_2$, 则点 $P$ 的横坐标为\lines.%120度怎么样?椭圆与另一个曲线联立是常有的

    \question[*]动圆$M$与圆$C_1:(x+1)^2+y^2=36$内切,与圆$C_2:(x-1)^2+y^2=4$外切,求圆心$M$ 的轨迹方程。%要先猜出是什么图形,然后证明

    \question[2014福建 ]设$P,Q$分别为圆$x^2+(y-6)^2=2$ 和椭圆$\dfrac{x^2}{10}+y^2=1$ 上的点,则$P,Q$ 两点间的最大距离是

%    \subsection{更多}
%        \question[**] 椭圆的中心为点 $E(-1,0)$, 它的一个焦点为 $F(-3,0)$, 相应于焦点 $F$ 的准线方程为 $x=\frac{-7}{2}$, 则这个椭圆的方程是\lines
\section{综合题}
    %%焦点三角形面积公式,柯西公式,三角代换
    \question[2014湖北 ]已知$F_1,F_2$是椭圆和双曲线的公共焦点, $P$是他们的一个公共点,且 $\angle F_1PF_2=\dfrac{\pi}{3}$, 则椭圆和双曲线的离心率的倒数之和的最大值为
    %**
    \question 设$F_1,F_2$分别是椭圆$\dfrac{x^2}{4}+y^2=1$的左、右焦点.
        \quson 若$P$是第一象限内该椭圆上的一点,且$\overrightarrow{PF_1}\cdot \overrightarrow{PF_2}=-\dfrac{5}{4}$,求点$P$的坐标;
        \quson 设过定点$M(0,2)$的直线$l$与椭圆交于不同的两点$A,B$,且$\angle AOB$为锐角(其中$O$为原点),求直线$l$的斜率$k$的取值范围.
    %定点改为(3,0), 锐角改为钝角,所设方程要随之改变

    \question[作业]设$A,B$分别为椭圆$\dfrac{x^2}{a^2}+\dfrac{y^2}{b^2}=1(a>b>0)$的左、右顶点,$(1,\frac32)$为椭圆上一点,椭圆长半轴的长等于焦距.
        \quson 求椭圆的方程;
        \quson 设$P(4,x)(x\ne0)$, 若直线$AP$,$BP$分别与椭圆相交异于$A,B$的点$M,N$,求证:$\angle MBN $ 为钝角.

    \question[]设$P$是圆$x^2+y^2=25$上的动点,点$D$是$P$在$x$轴上的投影,$M$为$PD$上一点,且$|MD|=\dfrac45|PD|$.
        \quson 当$P$在圆上运动时,求点$M$的轨迹$C$的方程;
        \quson[(弦长公式)]求过点$(3,0)$且斜率为$\dfrac45$的直线被$C$所截线段的长度.

    %%***
    \question 已知中心在原点,焦点在$x$轴上的椭圆$C$的离心率为$\dfrac12$,且经过点$M(1,3/2)$.
        \quson 求椭圆$C$的方程;
        \quson 是否存在过点$P(2,1)$的直线$l_1$ 与椭圆$C$相交于不同的两点$A,B$,满足$\overrightarrow{PA}\cdot\overrightarrow{PB}={\overrightarrow{PM}}^2$?若存在,求出直线$l_1$的方程;若不存在,请说明理由.
    %认为它存在,试着把它求出来,如果可以求出,自然就存在了。

    %%****
    \question 已知椭圆$C:\dfrac{x^2}{a^2}+\dfrac{y^2}{b^2}=1(a>b>0)$的离心率为$\dfrac{\sqrt6}{3}$,短轴一个端点到右焦点的距离为$\sqrt3$.
        \quson 求椭圆$C$的方程.
          \quson 直线$l:y=x+1$与椭圆交于$A,B$两点,$P$为椭圆上一点,求$\triangle PAB$ 面积的最大值.
          \quson 在第二问的基础上求$\triangle AOB$的面积.

    %三角形面积的最大值
    \question 已知椭圆$\dfrac{x^2}{a^2}+\dfrac{y^2}{b^2}=1(a>b>0)$的左右焦点为$F_1,F_2$, 离心率$e=\dfrac{\sqrt{3}}{2}$, $P$是椭圆上的动点,$\triangle PF_1F_2$ 面积的最大值为$\sqrt3$.
        \quson 求椭圆的方程;
        \quson 若直线$l$与椭圆交于$A,B$两点,线段$AB$的中点为$M$,$O$为坐标原点,且$|OM|=1$,求$\triangle AOB$ 面积的最大值。

\section{抛物线}
  \question[2014全国新课标II卷:三角形面积,弦长公式或联立直线]
    设$F$为抛物线$C:y^2=3x$的焦点,过$F$且倾斜角为$30^{\circ}$的直线交于$C$于$A,B$两点,$O$为坐标原点,则$\triangle OAB$的面积为\brackets
    \choice{\dfrac{3\sqrt3}4}{\dfrac{9\sqrt3}8}{\dfrac{63}{32}}{\dfrac94}
    %D |OA||y_1-y_2| 通过联立直线算出|y_1-y_2|  也可以用公式AB=\frac{2p}{\sin^2\alpha}
 

\end{document}

