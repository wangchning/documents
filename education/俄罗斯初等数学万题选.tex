\documentclass{article}
\begin{document}
%数学模式下也有个边框盒子 \tbs{boxed}。
\section{集合}
  \question 写出集合$\Omega=\set{1,2,[1,2]}$的全部子集。
\section{Tabular}
  tabular 环境提供了最简单的表格功能。它用 \tbs{hline} 命令表示横线,|
表示竖线;用 \& 来分列,用 \tbs\tbs 来换行;每列可以采用居中、居左、居右等
横向对齐方式,分别用 l、c、r 来表示。



\section{集合}
  \subsection{集合的概念}
    一般地,我们把研究对象统称为\emph{元素},把一些元素组成的总体叫做\emph{集合},简称\emph{集}。

    我班的全体同学构成一集合,我国的所有省份、直辖市、自治区也构成一集合。

    任何集合的元素都是确定的,也就是说任何一个元素都可以判定在不在这个集合中,要么在,要么不在,二者必居其一。
    我班所有男生构成一集合,但我班身材较高的人不可以构成集合。

    集合常用大写字母如$AB,C,\cdots$标记,元素常用小写字母如$a,b,c,\cdots$标记。元素在集合中就说属于,用符号表示为$a\in A$, 元素不在集合中就说不属于,$a\not\in A$.

    判断以下列表述是否构成集合:
    (1)大于 1 小于 9 的偶数;
    (2)我国的小河流。

    集合用花括号标注,如 $A=\set{2,4,6,8}$. 集合的元素不能重复出现,每个元素在集合中都是唯一的。
    \marginpar{\kaishu\small 正如我班没有两个某某某。}
    集合与你书写的顺序,描述的方式无关,比如\set{2,4,6,8} 和 \set{4,6,8,2} 是同一个集合,和 \ses{大于 1 小于 9 的偶数} 也是同一个集合,和 \ses{大于 0 小于 10 的偶数} 还是同一个集合。

    这两个性质分别叫做 \emph{唯一性}和 \emph{无序性},加上前面介绍的 \emph{确定性}构成集合的三大性质。

    两个集合相等是说构成这两个集合的元素是完全相同的,与它的描述方式无关。

    数学中有几个常用的数集,\N=\ses{全体非负整数},
    \Z=\ses{全体整数},\Q=\ses{全体有理数},\R=\ses{全体实数},正整数集合也经常用到,
    记作$\N^*$或$\N_+$

    \bpics{0.6}\begin{verbatim}
        \begin{gather}
            a = b+c+d \\
            x = y+z
        \end{gather}\end{verbatim}
    \mpics{0.3}
        \begin{gather}
          a = b+c+d \\
          x = y+z
        \end{gather}
    \epics

  \subsection{自测题}
    \begin{tasks}(4)
      \task \N=\ses{全体非负整数}
      \task \Q=\ses{全体有理数}
      \task \R=\ses{全体实数}
      \task \Z=\ses{全体整数}
    \end{tasks}









\end{document}