\documentclass{article}
\title{有理数及其运算}
\author{王崇宁}
\maketitle
\begin{document}
\section{有理数}
\section{数轴}
\section{绝对值}
\section{有理数的加法}
\section{有理数的减法}
\section{有理数的加减混合运算}
\section{有理数的乘法}
  \subsection{“负负得正”举例}
    \begin{verse}
      话题1:\\
      朋友的朋友是朋友(即正正得正);\\
      朋友的敌人是敌人(即正负得负);\\
      敌人的朋友是敌人(即负正得负);\\
      敌人的敌人是朋友(即负负得正)。\\
    \end{verse}
    \begin{verse}
话题2:\\
好人有好报是好事(正正得正);\\
好人有坏报是坏事(正负得负);\\
坏人有好报是坏事(负正得负);\\
坏人有坏报是好事(负负得正)。
\end{verse}
——摘自 卜以楼 让数学教育的文化价值在教学中鲜活地流淌 中学数学杂志(初中版)2011(6)
\subsection{}
\section{有理数的除法}
  \subsection{除法有分配律吗?}
    试卷上小明做错了一道题,但小明很是纳闷,不知道为什么错了。他是这样做的:
      $$2\div(-1\frac14+\frac56-frac58)$$
    \begin{align}
      \mathtext{解:原式}&=2\div(-1\frac14)+2\div\frac56-2\div\frac58\\
      &=2\times(-\frac45)+2\times\frac56-2\times\frac85\\
      &=2\times(-\frac45+\frac65-\frac85)\\
      &=-\frac{12}5
    \end{align}
   你发现他错在哪里了吗?
  原来尽管有理数的除法可以转化为乘法,但除法没有相应的交换律、结合律、分配律。$(b+c+d)\div a=\frac ba+\frac ca+\frac da$,但$a\div(b+c+d)\neq\frac ab+\frac ac+\frac ad$。千万不能出错哦。请你帮小明给出正确解答。

  \subsection{提高练习}
    \numa 如果一个数的绝对值与这个数的商等于$-1$,则这个数是\brackets
      \choicex{正数}{负数}{非正}{非负}
    \numa 下列说法错误的是(    )
     \choicex{正数的倒数是正数}{负数的倒数是负数}{任何一个有理数$a$的倒数等于}{乘积为$-1$的两个有理数互为负倒数}
    \numa 计算: 
      \numaa$-3.5\times(\frac16-0.5)\times\frac37\div(-\frac12)$
      \numaa$(2\frac13-3\frac12+1\frac1{45})\div(-1\frac16)$

    \numa 若$\frac{ab}c<0$,$ac>0$,请判断$b$的符号。
    \numa 若$a,b$是有理数,且$ab\ne0$,试求$\frac{|a|}a+\frac{|b|}b$的值.
\section{有理数的乘方}
  \subsection{练习}
     \numa 在$-|-4|^3$,$-(-4)^3$,$(-4)^3$,$-4^3$,最大的数是\brackets
    \choice{-|-4|^3}{-(-4)^3}{(-4)^3}{-4^3}
     \numa 设a<0,则下列说法中正确的是\brackets
        \choicex{$a$的偶次方的偶次方是负数}{$a$的奇次方的偶次方是负数}{$a$的奇次方的奇次方是负数}{$a$的偶次方的奇次方是负数}
     \numa 如果$a\ne0$,那么下列各式中一定成立的是\brackets
        \choice{-a^4>0}{a^2-a>0}{a-a^2>0}{(-a)^2>0}
     \numa 已知$A=a+a^2+a^3+\cdots+a^{2002}$,若$a=1$,则$A$等于多少?若$a=-1$,则$A$等于多少?
     \numa 已知$|a-1|+(b+2)^2=0$,求$(a+b)^{1001}$的值.
\section{科学记数法}

  \subsection{记数法的故事}
    古希腊的著名数学家、科学家阿基米德也列出了一种大数记法,是“亿”进位,亿,亿亿
  \subsection{实际问题}


    \numa 当你把纸对折一次时,可以得到$2$层;对折$2$次时,可以得到$4$层;对折$3$次时,可以得到$8$层;照这样折下去:
      \numaa 你能发现层数与折纸的次数的关系吗?
      \numaa 计算对折$5$次时层数是多少?
      \numaa 如果每张纸的厚度是$0.05$毫米,求对折$l0$次后纸的总厚度.
    \numa 意大利米兰国立歌舞剧场演出歌剧时,挪威电视台中转,猜一猜, 谁最早听到歌剧的开始?是与舞台相距$25$米的现场观众,还是距离$2900$千米的挪威电视观众?(声速是$340$米/秒,电波速度是$3×108$米/秒)
    \numa  一根方便筷子的长、宽、高大约为$0.5$cm、$0.4$cm、$20$cm,估计$1000$万双方便筷子要用多少木材?这些木材要砍伐半径为$0.1$米、高$10$米(除掉不可用的树稍)的大树多少棵?(保留三个有效数字) 
    \numa 计算机的存储容量的基本单位是字节,用$b$表示,计算机一般用Kb(千字节)或Mb(兆字节)或Gb(千兆字节)称为存储容量的计量单位,它们之间的关系为:1Kb=210 b ,1Mb=210  Kb,1Gb= 210Mb ,一种新款电脑的硬盘的存储容量为20Gb,它相当于多少Kb(用科学记数法)?
    \numa 将自然数1到15中的素数之积的相反数表示成科学记数法为\lines.


\section{有理数的混合运算}

\section{用计算器进行运算}
  \subsection{魔数153}
    任写一个数,它是3的倍数。把它的各个数字分别立方,把得到的立方数相加,得到一个新的数,再把新得到的数的每个数字分别立方,把得到的立方数相加,又得到一个新数。一直重复下去,你发现了什么?请借助计算器进行探索。

\section{}

\end{document}
