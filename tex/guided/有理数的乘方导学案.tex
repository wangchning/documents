\documentclass[twoside,12pt]{article}
\usepackage[includemp=true,marginparsep=.5cm,marginparwidth=3cm,left=1.7cm,right=.7cm,top=2cm,bottom=1.5cm]{geometry}
\input{guided_macros.tex}
\def\myp#1{\par\vspace*{1ex}\noindent{\heiti#1}\par}
\def\myskip{\par\vspace*{1in}}
\def\msection#1{{\heiti\numa~~#1}}
\def\qqquad{\quad\qquad}\def\qvuad{\hspace{4.5em}}
\begin{document}

\mytitle{有理数的乘方 第一课时}

\msection{学习目标}
(1)能够说出乘方的定义.\par
(2)能够把特定问题中的算式用乘方表示出来, 并正确计算简单的乘方.\\

\msection{定义}
\aside{乘方的这种表示方法为17世纪由数学家笛卡尔首先使用}
  \kaishu  
  为简便, 一般地, $n$个相同的因数$a$相乘, 记作\lines, 即

      \vspace*{-1em}
      $$a\times a\times\cdots\times a= \lines$$\par
      \vspace*{-.5em}
    这种求$n$个相同因数$a$的积的运算叫做乘方, 
    乘方的结果叫做\lines, $a$叫做\lines, $n$叫做\lines, $a^n$读作\lines[3]\text{(或\lines[3])}.
  \songti
\aside{$a$的$n$次方这种读法强调了表示$n$个$a$相乘这一运算过程;而$a$的$n$次幂这一读法更多强调的是这个运算的结果.}

\msection{课堂学习}
\noindent 


练习1: (1)$(-1.5)^1$ ;\qqquad (2)$\Dd {(\frac32)}^2$ ;\qqquad (3)$(-0.3)^3$ ;\qqquad (4)$(-3)^4$.
\myskip


练习2.$(1)\ (-2)^2$; \qvuad $(2)\ -2^4$; \qvuad $(3)\ -(-2)^4$.
\myskip

练习3.$(1)\ \Dd (-\frac34)^2$; \qvuad $(2)\ \Dd -\frac{3^2}4$; \qvuad $(3)\ \Dd -(\frac34)^2$.  
\myskip



\msection{小结}\par
a)乘方的定义 \par
b) 分数和负数作底数时,作为一个整体加上括号.\par
c) 乘方运算通常先化为乘法再进行计算.
\vspace{1em}

\msection{拔高训练}\par
  1.一个草履虫平均每经过$27$小时就会分裂成两个.假如以这些方式衍生的草履虫都能存活下来, $27$天之后, 这个草履虫及其后代共有多少个? (用$x^y$表示出来即可.)
\vspace{1em}

2.计算:(1)$-1.5^2$\qvuad (2)$\Dd -\frac{(-2)^4}{4}$ \qvuad (3)$\Dd -{(-1\frac12)}^3$
\end{document}
