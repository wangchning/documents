\documentclass[twoside,12pt]{article}
\usepackage[includemp=true,marginparsep=.5cm,marginparwidth=3cm,left=1.7cm,right=.7cm,top=2cm,bottom=1.5cm]{geometry}
\input{guided_macros.tex}
\def\myp#1{\par\vspace*{1ex}\noindent{\heiti#1}\par}
\def\myskip{\par\vspace*{.7in}}

\begin{document}

\mytitle{平面向量的数量积导学案 第一课时}

\begin{itemize*}
  \item 会使用平面向量的数量积的定义及其几何意义
  \item 能说出平面向量数量积的重要性质及运算律
  \item 了解用平面向量的数量积可以处理有关长度、角度和垂直的问题
\end{itemize*}


\section{数量积的物理背景}
在初中物理上,功等于\textsl{力与位移的乘积}. 这个定义仅适用于力与位移在同一个方向上时. 在高中,功被定义为\textsl{力和物体在力的方向上的位移的乘积},计算公式为 $W=Fs\cos\theta$.\aside{向上的分力不做功,就像水蒸气一样蒸发了}\par \includegraphics[width=3.5in]{d241-01.png}
\aside{ 这是不是意味着两个向量的乘积是一个数量?\\答:两个向量的数量积(也叫内积)是一个数量. }
力是一个向量,位移也是一个向量,而力在物体位移的方向上所做和功是一个标量,这可以说是我们定义两个向量的数量积的出发点.

 
\section{数量积的定义}
\aside{ 为什么叫数量积呢?叫点乘积岂不更好!因为两个向量运算的结果是一个数,所以叫数量积.} 
\begin{definition}向量$\veca,\vecb$的数量积$$\veca\cdot\vecb=|\veca||\vecb|\cos\theta$$
其中$\theta$是$\veca$与$\vecb$的夹角. 我们规定,零向量与任一向量的数量积为$0$
\end{definition}\aside{ 为什么要加一条规定呢?只是因为零向量与任意向量的夹角任意,没有办法计算$\cos\theta$.}

%\includegraphics[width=3.3in]{d241-02.png}

\textbf{注意}:中间的“$\cdot$”不可以省略,也不可以写成“$\times$”.


例1.已知$|\veca|=5$, $|\vecb|=4$,$\veca$与$\vecb$的夹角$\theta=120^\circ$,求$\veca\cdot\vecb$.\vspace{2ex}

 
练习1.已知$|\vecp|=8$, $|\vecq|=6$,  向量$\vecp$ 和 $\vecq$的夹角是$60^\circ$, 求 $\vecp\cdot\vecq$.

\myskip
练习2.设$|\veca|=12$,$|\vecb|=9$, $\veca\cdot\vecb=−54$,求向量$\veca$和$\vecb$的夹角$\theta$.
\myskip 

\section{数量积的性质}
\myp{思考}
1.向量的数量积什么时候为正,什么时候为负?
\aside{ 老师,你问的要是非零向量就好了. }
\myskip
   
2.$|\veca\cdot\vecb|$与$|\veca||\vecb|$之间有不等关系吗?
\aside{ 反正不是相等关系.}
\myskip

\myp{数量积的性质}\vspace{1ex}
\hrule\vspace{1ex}
  设$\veca,\vecb$都是非零向量, 
$\theta$是$\veca$与$\vecb$的夹角,则
  \begin{itemize*}
    \item $\veca\perp\vecb\iff\veca\cdot\vecb=0$\quad(判断两向量垂直的依据)
    \item 当$\veca$与$\vecb$同向时,$\veca\cdot\vecb=|\veca||\vecb|$,
     当$\veca$与$\vecb$反向时, $\veca\cdot\vecb=−|\veca||\vecb|$.特别地,$\veca\cdot\veca=|\veca|^2$或$|\veca|=\sqrt{\veca\cdot\veca}$\quad(用于计算向量的模)
    \item $|\veca\cdot\vecb|\le|\veca|\cdot|\vecb|$\quad 一个常用不等式
  \end{itemize*}
\hrule\vspace{1ex}

例2.判断正误\begin{itemize*}\def\mybra{\quad(\quad)}
  \item 若$\veca=\veczero$,则对任一向量$\vecb$,有$\veca\cdot\vecb=0$.\mybra
  \item 若$\veca\ne\veczero,\vecb\ne\veczero$,则有$\veca\cdot\vecb\ne0$.\mybra
  \item 若$\veca\ne\veczero, \veca\cdot\vecb=0$,则 $\vecb=\veczero$.\mybra
  \item 若$\veca\cdot\vecb=0$,则$\veca,\vecb$中至少有一 个 为 $\veczero$.\mybra
  \item 若$\vecb\ne\veczero$,且$\veca\cdot\vecb=\vecb\cdot\vecc$,则 $\veca=\vecc$.\mybra
  \item 对任意向量$\veca$ ,有$\veca^2=|\veca|^2$.\mybra \aside{ 符号$\veca^2$尚未定义,这里假定聪明的你明白是什么意思}
\end{itemize*}
\vfill
\section{数量积的几何意义}
平面向量数量积 $\veca\cdot\vecb$的几何意义

定义$\veca\cdot\vecb=|\veca||\vecb|\cos\theta$中,$|\veca|\cos\theta$叫做向量$\veca$在$\vecb$上的\textbf{投影},$|\vecb|\cos\theta$叫做向量$\vecb$在$\veca$上的\textbf{投影}. 

\includegraphics[width=5.5in]{d241-03.png}

向量 $\veca$ 与$\vecb$ 的数量积等于$\veca$ 的长度 $|\veca|$ 与$\vecb$ 在$\veca$ 的方向上的投影$|\vecb|\cos\theta$的积.


\section{运算律}
\myp{数量积运算律}\hrule\vspace{1ex}
  \begin{enumerate*}
    \item $\veca\cdot\vecb=\veca\cdot\vecb$\quad 交换律
    \item $(\lambda\veca)\cdot\vecb=\lambda(\veca\cdot\vecb)=\veca\cdot(\lambda\vecb)$ \quad 与数的结合律
    \item $(\veca+\vecb)\cdot\vecc=\veca\cdot\vecc+\vecb\cdot\vecc$\quad 分配律
\aside{ 学了数量积的坐标表示再来证明分配律,会容易得多.}
  \end{enumerate*}
\hrule\vspace{1ex}
    \par 思考:\aside{ 老师,放这里是分明地暗示不成立呀!}$(\veca\cdot\vecb)\cdot\vecc=\veca\cdot(\vecb\cdot\vecc)$ 成立吗?为什么?
\myskip


 
例3.求证:(1)$(\veca +\vecb)^2=\veca^2+2\veca\cdot\vecb+\vecb^2$\par     
(2)$(\veca+\vecb)\cdot(\veca-\vecb)=\veca^2-\vecb^2$
\myskip
  
例4.已知$|\veca|=6,|\vecb|=4$,$\veca$与$\vecb$的夹角为$60^\circ$,求$(\veca+2\vecb)\cdot(\veca-3\vecb)$.
\newpage

\section{数量积的几点应用}
\subsection{判断两个向量是否垂直}
例5.已知$|\veca|=3, |\vecb|=4$(且$\veca$与$\vecb$不共线),且仅当$k$为何值时,向量$\veca+k\vecb$与$\veca-k\vecb$互相垂直?\aside{边注这么多空白,不拿来演草岂不可惜.}
\myskip
变式训练1:若向量$\veca,\vecb,\vecc$满足$\veca\sslash\vecb$且$\veca\perp\vecc$,
则$\vecc\cdot(\veca+2\vecb)=$\lines.
\subsection{求向量之间的夹角}
例6.设$\veca,\vecb$是两个不共线的非零向量,且$|\veca|=|\vecb|=|\veca-\vecb|$,求向量$\veca$与$\veca+\vecb$的夹角
\myskip
变式训练2:已知$|\veca|=|\vecb|=2$, $(\veca+2\vecb)\cdot(\veca-\vecb)=-2$, 则$\veca$与$\vecb$的夹角为\lines.
\myskip

\section{小结}
\begin{itemize*}
  \item 向量的数量积的物理模型是力的做功.
  \item $\veca\cdot\vecb$的结果是个数量.
  \item 数量积满足交换律,分配律,与数
的结合律,但是不满足结合律
  \item 利用数量积可以求两向量的夹角,特别是可以判定垂直.
\end{itemize*}
变式训练答案:1.\underline{~0~} 2.\underline{~$\pi/3$~}

\end{document}
