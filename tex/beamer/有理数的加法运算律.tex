\documentclass{beamer}
\usepackage{ctex}
\usepackage{graphicx,booktabs}
\usepackage{tikz}
\usepackage{comment}
\usepackage{multirow}
\usetikzlibrary{arrows,arrows.spaced,arrows.meta,calc,intersections,through,backgrounds,math,angles,shapes}
\usetheme{Boadilla}
\beamersetaveragebackground{black!6}
\def\mathfamilydefault{\rmdefault}
\setbeamersize{text margin left=1.3cm}\setbeamersize{text margin right=1.2cm}
\newcommand\mytitle[1]{\title{#1}\author{王崇宁}\institute[四中]{郑州四中}\logo{\includegraphics[width=1cm]{cows.jpg}}\maketitle\kaishu\Large}
\newcommand\colorwordsa[1]{\textcolor[rgb]{0,.63,.91}{\heiti #1}}
\newcommand\colorwordsb[1]{\textcolor[rgb]{.75,.53,.21}{\heiti #1}}
\setbeamercolor{mycolor}{fg=black,bg=pink}
\setbeamercolor{myupcol}{fg=white,bg=purple}
\setbeamercolor{mylowcol}{fg=black,bg=pink}
\newcommand\beamercolortextbox[1]{\settowidth\lengtha{#1}\hspace{.5em}\raisebox{-.6ex}{\begin{beamercolorbox}[rounded=true,shadow=true,wd=\lengtha,colsep*=-.4ex]{mycolor}#1\end{beamercolorbox}}}
\newcommand\beamercolorlinebox[1]{\settowidth\lengtha{#1}\centerline{\begin{beamercolorbox}[rounded=true,shadow=true,wd=\lengtha]{mycolor}#1\end{beamercolorbox}}}\newlength\lengtha
\newcommand\threesides{\begin{picture}(14,12)\put(0,0){\qbezier(0,0)(0,8)(7,12)\qbezier(7,12)(14,8)(14,0)\qbezier(14,0)(7,-4)(0,0)}\end{picture}}


\def\displaygray{\setbeamercovered{transparent}}
\newcommand*\circled[1]{\hspace{1pt}\tikz[baseline=(char.base)]\node[very thin,shape=circle,draw,inner sep=.5pt,minimum size=2pt](char){#1};\hspace{1pt}}%



\newcommand\Dd{\displaystyle}\newcommand{\Tt}{\textstyle}\newcommand{\Ss}{\scriptstyle}\newcommand{\Sss}{\scriptscriptstyle}%

\setlength\mathsurround{0.5ex}%文本公式前后留下空白
\makeatletter
\renewcommand\normalsize{\@setfontsize\normalsize\@xpt\@xiipt\abovedisplayskip8\p@\@plus3\p@\@minus5\p@\abovedisplayshortskip\z@\@plus3\p@\belowdisplayshortskip5\p@\@plus3\p@\@minus7\p@\belowdisplayskip\abovedisplayskip\let\@listi\@listI}%行间公式上下
\@addtoreset{equation}{section}
\renewcommand\theequation{\oldstylenums{\thesection}.\oldstylenums{\arabic{equation}}}%公式按章节编号
\makeatother
%\setlength\mathsurround{0.ex}%数学模式与文本模式混排时留出的间距

\newcommand\an[1][a]{\ensuremath{\{#1_n\}}}%sequence
\newcommand\ud{\mathrm{d}}
\newcommand{\ue}{\mathrm{e}}%正体字母
\newcommand\triabc{\ensuremath{\triangle ABC}}%三角形ABC
\newcommand\cnm[2][n]{\ensuremath{\textrm{C}^{#2}_{#1}}}%组合数
%数学题目编辑
\newcommand\lines[1][1.2]{\,\underline{\mbox{\hspace{#1cm}}}\,}% 填空题的横线
\newcommand\brackets[1][2]{\nolinebreak\hfill\mbox{~(\hspace{#1em})}\\}% 选择题的括号
%扩展命令
\newcommand\qqquad{\qquad\quad}
\def\aside#1{\marginpar{\footnotesize #1}}
%自动编号之最简
\newcommand\numa{\refstepcounter{numi}\thenumi}\newcounter{numi}
\newcommand\numb{\refstepcounter{numii}\thenumii}\newcounter{numii}
\newcommand\numc{\refstepcounter{numiii}\thenumiii}\newcounter{numiii}
\newcommand\numaa{\refstepcounter{numai}\thenumai}\newcounter{numai}[numi]

\newcommand{\question}[1][]{\par\vspace{1ex}\noindent\refstepcounter{numberi}\textbf{\thenumberi.}\ensuremath{#1}}\newcounter{numberi}[subsection]% 每道小题自动编号
\newcommand\quson{\\ \hspace*{1em}\refstepcounter{numberii}\thenumberii)~}\newcounter{numberii}[numberi]% 每道小题自动编号
\newcommand\choice[5][4]{\vspace*{-1em}\begin{tasks}(#1)\task$#2$\task$#3$\task$#4$\task$#5$\end{tasks}\vspace*{-1.2em}}%选择题排版之数学模式
\newcommand\choicex[5][4]{\vspace*{-1em}\begin{tasks}(#1)\task#2\task#3\task#4\task#5\end{tasks}\vspace*{-1.2em}}%选择题排版之yysg模式
\newcommand\tbs[1][]{\texttt{\char92#1}}

\newcommand\set[1]{\lbrace\ensuremath{#1}\rbrace}
\newcommand\setx[1]{\{#1\}}

\newcommand\mybf[1]{{\bm#1}}
\def\bfR{\mybf R}
\def\bfN{\mybf N}
\def\bfZ{\mybf Z}
\def\bfQ{\mybf Q}
\def\bfC{\mybf C}
\def\bfZp{\mybf Z^+}

\newcommand\myvec[1]{\bm#1}
\def\veca{\myvec a}
\def\vecb{\myvec b}
\def\vecc{\myvec c}
\def\vecd{\myvec d}
\def\vece{\myvec e}
\def\vecf{\myvec f}
\def\vecs{\myvec e}
\def\vecp{\myvec p}
\def\vecq{\myvec q}
\def\veci{\myvec i}
\def\vecj{\myvec j}
\def\vecei{\myvec{e_1}}
\def\veceii{\myvec{e_2}}
\def\veczero{\myvec 0}
\newcommand\lvec[1]{\overrightarrow{#1}}





\begin{document}
\kaishu\Large
\def\hspacea{\hspace*{1.5in}}
\mytitle{有理数的加法运算律}
\begin{frame}{引入}
  \begin{exampleblock}{你会用简便方法计算吗?}
    $$31+(-28)+28+69$$
  \end{exampleblock}
\end{frame}

\begin{frame}{小学学过的交换律和结合律}
  \begin{block}{加法交换律}
    两个数相加, 交换加数的位置, 和不变. 
  \end{block}
  \begin{alertblock}<2>{加法结合律}
    三个数相加, 先把前两个数相加, 或者先把后两个数相加, 和不变. 
  \end{alertblock}
\end{frame}

\begin{frame}{交换律与结合律}
  \begin{block}{穿衣服}
    \begin{itemize}[<+->]
      \item 先穿背心再穿夹克, 跟先穿夹克再穿背心不一样, 不满足\colorwordsa{交换律};
      \item 先穿背心穿夹克, 再穿羽容服, 跟先穿背心, 再穿夹克穿羽容服是一样的, 满足\colorwordsa{结合律}.
    \end{itemize}
  \end{block}
  \begin{exampleblock}<3>{烤鱼}
   \hspacea 盐$\Psi$鱼$\Psi$火  
  \end{exampleblock}
\end{frame}


\begin{frame}{检验有理数加法的交换律}
  \begin{block}{有负数参与的加法}
    $$(-8)+(-9)\quad (-9)+(-8)$$
    $$4+(-7)\quad (-7)+4$$
  \end{block}
  \begin{alertblock}<2>{有理数的加法交换律}
    两个有理数相加, 交换加数的位置, 和不变.
    $$a+b=b+a$$
  \end{alertblock}
\end{frame}

\begin{frame}{检验有理数加法的结合律}
  \begin{block}{有负数参与的加法}
    $$[2+(-3)]+(-8)\quad 2+[(-3)+(-8)]$$
    $$[10+(-10)]+(-5)\quad 10+[(-10)+(-5)]$$
  \end{block}
  \begin{alertblock}<2>{有理数的加法结合律}
    三个有理数相加, 先把前两个数相加,或者先把后两个数相加, 和不变.
    $$(a+b)+c=a+(b+c)$$
  \end{alertblock}
\end{frame}

\begin{frame}{扩展}
  \begin{block}{加法交换律和结合律}
    \colorwordsb{加法交换律:} 若干个有理数相加, 可以任意交换加数的位置, 和不变.\par
    \colorwordsb{加法结合律:} 若干个有理数相加, 可以对任意几个加数加上括号, 和不变.
  \end{block}
\end{frame}

\begin{frame}{计算}
  \begin{exampleblock}{练习1. 计算下列各题}
    (1)$(-3)+40+(-32)+(-8)$;\quad(2)$13+(-56)+47+(-34)$;\quad(3)$43+(-77)+27+(-43)$.
  \end{exampleblock}
\end{frame}

\begin{frame}{实际问题}
  \begin{exampleblock}{练习2.}
    奶牛阿草有$18$块石头, 被奶牛阿花借走$15$块, 然后又被管理员要走$8$块, 请问现在奶牛阿草有几块石头?
  \end{exampleblock}
\end{frame}

\begin{frame}{实际问题}
  \begin{exampleblock}{练习3.}
    有一批食品罐头, 标准质量为每听$454$g. 现抽取$10$听样品进行检测, 结果如下表:
    \begin{table}[htp]
      \centering
      %\caption{}
      \begin{tabular}{|c|c|c|c|c|c|}
        \toprule[1pt]
        听号&1&2&3&4&5\\
       质量/g &444&459&454&459&454\\
        \toprule[1pt]
        听号&6&7&8&9&10\\
        质量/g&454&449&454&459&464\\
        \toprule[1pt]
      \end{tabular}
    \end{table}
  \end{exampleblock}
\end{frame}


\begin{frame}{今日作业}
 \LARGE 1. A1本\underline{21, 22}两页, \par
      2.课本习题2.5的\underline{1,4}两题, 写在A1本第23页.
\end{frame}

\end{document}
