\documentclass{articla}
\usepackage{ctex}
\begin{document}
\def\moli{\par$\triangle$\par}
\section{不等式}
\moli
已知6枝玫瑰与3枝康乃馨的价格之和大于24,而4枝攻瑰与5枝康乃馨的价格之和小于22元,则2枝玫瑰的价格和3枝康乃馨的价格比较,结果是(  ).

 A.2枝玫瑰价格高  B.3枝康乃馨价格高     C.价格相同     D.不确定

解:
2支玫瑰 1支康乃馨 大于8

4支玫瑰 2支康乃馨 大于16

4支玫瑰 5支康乃馨 小于22

3支康乃馨 小于6

1支康乃馨 小于2

2支玫瑰 大于6

2支玫瑰 大于 3支康乃馨



\moli


已知$x,y$都在区间$(-2,2)$内, 且$xy=1$, 则函数$u=\frac{4}{4-x^2}+\frac9{9-x^2}$的最小值是

A$\frac85$  B$\frac{24}{11}$ C$\frac{12}7$ D$\frac{12}5$

已知$x,y$在区间$(0,2)$内且$xy=1$,则函数$u=\frac4{4-x^2}+\frac9{9-y^2}$的最小值是

已知$a,b$分别在区间$(0,1),(0,\frac23)$内,且$ab=\frac16$, 则函数$u=\frac1{1-a^2}+\frac1{1-b^2}$的最小值

已知$a,b$在区间$(0,1)$内,且$ab=\frac16$, 则函数$u=\frac1{1-a^2}+\frac1{1-b^2}$的最小值(求出结果需验证$b$)

已知$m,n$在区间$(0,1)$内, 且$mn=\frac1{36}$, 则函数$u=\frac1{1-m}+\frac1{1-n}=\frac{2-(m+n)}{1-(m+n)+\frac1{36}}=1+\frac{35}{37-36(m+n)}$当$m+n$取最小值时取到最小值.

已知$m,n$在区间$(0,1)$内, 且$mn=\frac1{36}$, 则$m+n$的最小值是

$(m+n)^2=\frac19+(m-n)^2$

所以$m=n=\frac16$时取得最小值$\frac{12}5$




\moli


不等式$\sqrt{\log_2x-1}+\frac12\log_{\frac12}x^3+2>0$的解集为

A[2,3)  B(2,3] C[2,4) D(2,4]

$$\sqrt{\log_23-1}-\frac32\log_2x+2>0$$
$$\sqrt{a-1}-\frac32a+2>0$$
$$\sqrt{a-1}-\frac32(a-1)+\frac12$$
$$m-\frac32m^2+\frac12>0$$
$$3m^2-2m-1<0$$
$$0\le m<1$$
$$1\le a=\log_2x<2$$
$$2\le x<4$$


\moli
使关于$x$的不等式$\sqrt{x-3}+\sqrt{6-x}\ge k$有解的实数$k$的最大值是

$\sqrt{x-3}+\sqrt{6-x}\ge k$的最大值点($3\le x\le6)

平方

$\sqrt6$

\moli





\end{document}

