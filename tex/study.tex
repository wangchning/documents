\documentclass{article}
\usepackage{verbatim,bm,booktabs}
\usepackage{amsmath}
\usepackage[UTF8,space]{ctex}
\usepackage[CJKbookmarks=true]{hyperref}
\usepackage[margin=3.5cm]{geometry}
\begin{document}
\section{字符编码}
    \subsection{ASCII编码}
        使用7位(bits)表示一个字符,共128字符;但是7位编码的字符集只能支持128个字符,ASCII 扩展字符集使用8位(bits)表示一个字符,共256字符
    \begin{table}[htbp]
  \centering
  \caption{ASCII}
    \begin{tabular}{cl|cc|cc|cc}
    \toprule
    0     & 空字符    & 33    & \char33    & 65    & A        & 97    & a \\
    1     & 标题开始  & 34    & \char34    & 66    & B        & 98    & b \\
    2     & 正文开始  & 35    & \char35    & 67    & C        & 99    & c \\
    3     & 正文结束  & 36    & \char36    & 68    & D        & 100   & d \\
    4     & 传输结束  & 37    & \char37    & 69    & E        & 101   & e \\
    5     & 请求      & 38    & \char38    & 70    & F        & 102   & f \\
    6     & 收到通知  & 39    & \char39    & 71    & G        & 103   & g \\
    7     & 响铃      & 40    & \char40    & 72    & H        & 104   & h \\
    8     & 退格      & 41    & \char41    & 73    & I        & 105   & i \\
    9     & 水平制表符& 42    & \char42    & 74    & J        & 106   & j \\
    10    & 换行键    & 43    & \char43    & 75    & K        & 107   & k \\
    11    & 垂直制表符& 44    & \char44    & 76    & L        & 108   & l \\
    12    & 换页键    & 45    & \char45    & 77    & M        & 109   & m \\
    13    & 回车键    & 46    & \char46    & 78    & N        & 110   & n \\
    14    & 不用切换  & 47    & \char47    & 79    & O        & 111   & o \\
    15    & 启用切换  & 48    & 0          & 80    & P        & 112   & p \\
    16  & 数据链路转义& 49    & 1          & 81    & Q        & 113   & q \\
    17    & 设备控制1 & 50    & 2          & 82    & R        & 114   & r \\
    18    & 设备控制2 & 51    & 3          & 83    & S        & 115   & s \\
    19    & 设备控制3 & 52    & 4          & 84    & T        & 116   & t \\
    20    & 设备控制4 & 53    & 5          & 85    & U        & 117   & u \\
    21    & 拒绝接收  & 54    & 6          & 86    & V        & 118   & v \\
    22    & 同步空闲  & 55    & 7          & 87    & W        & 119   & w \\
    23    & 传输块结束& 56    & 8          & 88    & X        & 120   & x \\
    24    & 取消      & 57    & 9          & 89    & Y        & 121   & y \\
    25    & 介质中断  & 58    & \char58    & 90    & Z        & 122   & z \\
    26    & 替补      & 59    & \char59    & 91    & \char91  & 123   & \char123 \\
    27    & 溢出      & 60    & \char60    & 92    & \char92  & 124   & \char124 \\
    28    & 文件分割符& 61    & \char61    & 93    & \char93  & 125   & \char125 \\
    29    & 分组符    & 62    & \char62    & 94    & \char94  & 126   & \char126 \\
    30    & 记录分离符& 63    & \char63    & 95    & \char95  & 127   & \char127 \\
    31    & 单元分隔符& 64    & \char64    & 96    & \char96  &       &  \\
    32    & Space     &       &       &       &       &       &  \\
    \bottomrule
    \end{tabular}%
  \label{ascii}%
\end{table}%

    \subsection{GB2312}
        国内专家把那些127号之后的奇异符号们(即EASCII)取消掉,规定:一个小于127的字符的意义与原来相同,但两个大于127的字符连在一起时,就表示一个汉字,前面的一个字节(他称之为高字节)从0xA1用到 0xF7,后面一个字节(低字节)从0xA1到0xFE,这样我们就可以组合出大约7000多个简体汉字了。在这些编码里,还把数学符号、罗马希腊的 字母、日文的假名们都编进去了,连在ASCII里本来就有的数字、标点、字母都统统重新编了两个字节长的编码,这就是常说的"全角"字符,而原来在127号以下的那些就叫"半角"字符了。
        (???为什么从0xA1开始呢? 其它的去哪儿了? )

    \subsection{GB 18030}
        是中华人民共和国现时最新的内码字集,与GB 2312-1980完全兼容,与GBK基本兼容,支持GB 13000及Unicode的全部统一汉字,共收录汉字70244个。GB 18030主要有以下特点:
        \begin{itemize}
            \item 与UTF-8相同,采用多字节编码,每个字可以由1 个、2个或4个字节组成。
            \item 编码空间庞大,最多可定义161万个字符。
            \item 支持中国国内少数民族的文字,不需要动用造字区。
            \item 汉字收录范围包含繁体汉字以及日韩汉字
        \end{itemize}

    \subsection{Unicode}
        在计算机科学领域中,Unicode(统一码、万国码、单一码、标准万国码)是业界的一种标准,它可以使电脑得以体现世界上数十种文字的系统。Unicode 是基于通用字符集(Universal Character Set)的标准来发展,并且同时也以书本的形式[1]对外发表。Unicode 还不断在扩增, 每个新版本插入更多新的字符。直至目前为止的第六版,Unicode 就已经包含了超过十万个字符(在2005 年,Unicode 的第十万个字符被采纳且认可成为标准之一)、一组可用以作为视觉参考的代码图表、一套编码方法与一组标准字符编码、一套包含了上标字、下标字等字符特性的枚举等。Unicode 组织(The Unicode Consortium)是由一个非营利性的机构所运作,并主导 Unicode 的后续发展,其目标在于:将既有的字符编码方案以Unicode 编码方案来加以取代,特别是既有的方案在多语环境下,皆仅有有限的空间以及不兼容的问题。

    \subsubsection{UTF-32}
        上述使用4字节的数字来表达每个字母、符号,或者表意文字(ideograph),每个数字代表唯一的至少在某种语言中使用的符号的编码方案,称为UTF-32。UTF-32又称UCS-4是一种将Unicode 字符编码的协定,对每个字符都使用4字节。就空间而言,是非常没有效率的。

    \subsubsection{UTF-16}
        尽管有Unicode字符非常多,但是实际上大多数人不会用到超过前65535个以外的字符。因此,就有了另外一种Unicode编码方式,叫做UTF-16(因为16位 = 2字节)。UTF-16将0–65535范围内的字符编码成2个字节,如果真的需要表达那些很少使用的"星芒层(astral plane)" 内超过这65535范围的Unicode字符,则需要使用一些诡异的技巧来实现。UTF-16编码最明显的优点是它在空间效率上比UTF-32高两倍,因为每个字符只需要2个字节来存储(除去65535范围以外的),而不是UTF-32中的4个字节。并且,如果我们假设某个字符串不包含任何星芒层中的字符,那么我们依然可以在常数时间内找到其中的第N个字符,直到它不成立为止这总是一个不错的推断。

    \subsubsection{UTF-8}
        UTF-8(8-bit Unicode Transformation Format)是一种针对Unicode的可变长度字符编码(定长码),也是一种前缀码。它可以用来表示Unicode标准中的任何字符,且其编码中的第一个字节仍与ASCII兼容,这使得原来处理ASCII字符的软件无须或只须做少部份修改,即可继续使用。因此,它逐渐成为电子邮件、网页及其他存储或传送文字的应用中,优先采用的编码。互联网工程工作小组(IETF)要求所有互联网协议都必须支持UTF-8编码。

\section{编程该使用哪种字体}
    在 Google 里搜了一下,出现最多的是 Courier New 。比较一下,与 WinEdt9 中使用的相差不大,这里是我最喜欢的字体。
\section{海伦三角形数据处理记}
    编写 Python 程序,发现不能运行,原因是 2a 应该写成 2*a 。

    条件是 $a\le b\le c$ , $a+b>c$ 且 $a+b+c=n$. 从中可以解出
    $$1\le a\le \big[n/3\big]$$
    确定 $a$ 之后,又可以解出 $b$ 的范围
    $$\max\bigg(a, \bigg[\dfrac{n-2a}{2}\bigg]\bigg)\le b\le\frac{n-a}{2}+1$$
    然后可以确定 $c$. 先是把 $b$ 的下界写成了 $a$, 后来改成了 $[(n-2a)/2]+1$, 发现算出来的结果有重复的,最后才改正确。

    输出的时候带 .0, 用了 int 函数才变回整数。判断一个数是不是整数是在这里想出来的。

    将要把产生的海伦三角形输出到 Excel 中,用了 IDLE 另存为 txt, 然后利用逗号分隔符导入。先是把一堆三角形三边的长混在一起,但是在 Excel 中没有找到办法分开。后来先导出 $a$, 又导出 $b$ 和 $c$, 在 Excel 中算出周长与面积来。

    在 Excel 中打开 txt 文件,注意选择所有文件类型才可以找到。下面可以选择使用 Tab 键、分号、逗号、空格或其它可以写在框框里的符号。

    照原先的想法,要找出周长和面积都相等的两个或多个三角形。因为周长相等的都在一块,所以用上面的减下面的,如果周长和面积都是 0, 那么有两个周长和面积都相等的三角形,如此就可以找到所有这样的三角形。(可是,我错了!)如果都是 0 那么它们的差也是 0,这样我就可以把它们筛选出来。可惜,前面犯了错误!

\end{document}

