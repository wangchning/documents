\documentclass{article}
\usepackage[UTF8]{ctex}
\setlength{\mathsurround}{0.25em}
\begin{document}
\section{ISBN码}
ISBN(International Standard Book Number)由10个数字连成,它总是一个特殊的数,把它作为一个向量,与向量$(1,2,\cdots,10)$的内积是11的倍数。这样,错写一个数或对调两个不相等的数就不再是$11$的倍数,从而可以被检验出一来。
\section{皮卡定理}
\heiti 皮卡定理:\songti 格点多边形的面积等于 I+B/2-1,I 为其内部格点数,B 为边界上的格点数。\par
\heiti 证:\songti 两个满足皮卡定理的格点多边形拼接在一起也满足皮卡定理。因为单位正方形满足皮卡定理,所以矩形都满足皮卡定理。矩形沿对角线划为两个全等的直角三角形,这两个三角形用皮卡公式算出相同的值,由皮卡公式的可加性就得到直角三角形满足皮卡定理。任何一个格点三角形都可以由矩形及直角三角形得到,因而都满足皮卡定理。三角形可以拼接成任意多边形,所以皮卡定理成立。
\section{牛顿法}
在计算机上实现除法不便使用试商的办法,因而考虑把除法转换为乘法。用牛顿法计算$1/b$。求$f(x)= b-1/x$的零点。如果$x_0$充分靠近$f(x)$的零点,那么过$x_0$的切线与$x$轴的交点将更接近$f(x)$的零点。这样就充许我们用迭代的方法求零点的近似解。不难推出牛顿迭代公式$$x_{n+1}=x_n-\frac{f(x_n)}{f'(x_n)}.$$就这个问题而言,$$x_{n+1}=x_n(2-bx_n),$$而且$$Rel(x_{n+1})=b|x{n+1}-\frac1b|=|bx_n(2-bx_n)-1|=b^2(x_n-\frac1b)^2=[Rel(x_n)]^2.$$要使$x_n$收敛到$1/b$,就要使$|bx_0-1|<1$,即$0<x_0<2/b$。
\end{document}
